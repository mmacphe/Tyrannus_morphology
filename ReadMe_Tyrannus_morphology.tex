\documentclass[]{article}
\usepackage{lmodern}
\usepackage{amssymb,amsmath}
\usepackage{ifxetex,ifluatex}
\usepackage{fixltx2e} % provides \textsubscript
\ifnum 0\ifxetex 1\fi\ifluatex 1\fi=0 % if pdftex
  \usepackage[T1]{fontenc}
  \usepackage[utf8]{inputenc}
\else % if luatex or xelatex
  \ifxetex
    \usepackage{mathspec}
  \else
    \usepackage{fontspec}
  \fi
  \defaultfontfeatures{Ligatures=TeX,Scale=MatchLowercase}
\fi
% use upquote if available, for straight quotes in verbatim environments
\IfFileExists{upquote.sty}{\usepackage{upquote}}{}
% use microtype if available
\IfFileExists{microtype.sty}{%
\usepackage[]{microtype}
\UseMicrotypeSet[protrusion]{basicmath} % disable protrusion for tt fonts
}{}
\PassOptionsToPackage{hyphens}{url} % url is loaded by hyperref
\usepackage[unicode=true]{hyperref}
\hypersetup{
            pdftitle={Morphology\_Markdown},
            pdfauthor={Maggie P. MacPherson},
            pdfborder={0 0 0},
            breaklinks=true}
\urlstyle{same}  % don't use monospace font for urls
\usepackage[margin=1in]{geometry}
\usepackage{color}
\usepackage{fancyvrb}
\newcommand{\VerbBar}{|}
\newcommand{\VERB}{\Verb[commandchars=\\\{\}]}
\DefineVerbatimEnvironment{Highlighting}{Verbatim}{commandchars=\\\{\}}
% Add ',fontsize=\small' for more characters per line
\usepackage{framed}
\definecolor{shadecolor}{RGB}{248,248,248}
\newenvironment{Shaded}{\begin{snugshade}}{\end{snugshade}}
\newcommand{\KeywordTok}[1]{\textcolor[rgb]{0.13,0.29,0.53}{\textbf{#1}}}
\newcommand{\DataTypeTok}[1]{\textcolor[rgb]{0.13,0.29,0.53}{#1}}
\newcommand{\DecValTok}[1]{\textcolor[rgb]{0.00,0.00,0.81}{#1}}
\newcommand{\BaseNTok}[1]{\textcolor[rgb]{0.00,0.00,0.81}{#1}}
\newcommand{\FloatTok}[1]{\textcolor[rgb]{0.00,0.00,0.81}{#1}}
\newcommand{\ConstantTok}[1]{\textcolor[rgb]{0.00,0.00,0.00}{#1}}
\newcommand{\CharTok}[1]{\textcolor[rgb]{0.31,0.60,0.02}{#1}}
\newcommand{\SpecialCharTok}[1]{\textcolor[rgb]{0.00,0.00,0.00}{#1}}
\newcommand{\StringTok}[1]{\textcolor[rgb]{0.31,0.60,0.02}{#1}}
\newcommand{\VerbatimStringTok}[1]{\textcolor[rgb]{0.31,0.60,0.02}{#1}}
\newcommand{\SpecialStringTok}[1]{\textcolor[rgb]{0.31,0.60,0.02}{#1}}
\newcommand{\ImportTok}[1]{#1}
\newcommand{\CommentTok}[1]{\textcolor[rgb]{0.56,0.35,0.01}{\textit{#1}}}
\newcommand{\DocumentationTok}[1]{\textcolor[rgb]{0.56,0.35,0.01}{\textbf{\textit{#1}}}}
\newcommand{\AnnotationTok}[1]{\textcolor[rgb]{0.56,0.35,0.01}{\textbf{\textit{#1}}}}
\newcommand{\CommentVarTok}[1]{\textcolor[rgb]{0.56,0.35,0.01}{\textbf{\textit{#1}}}}
\newcommand{\OtherTok}[1]{\textcolor[rgb]{0.56,0.35,0.01}{#1}}
\newcommand{\FunctionTok}[1]{\textcolor[rgb]{0.00,0.00,0.00}{#1}}
\newcommand{\VariableTok}[1]{\textcolor[rgb]{0.00,0.00,0.00}{#1}}
\newcommand{\ControlFlowTok}[1]{\textcolor[rgb]{0.13,0.29,0.53}{\textbf{#1}}}
\newcommand{\OperatorTok}[1]{\textcolor[rgb]{0.81,0.36,0.00}{\textbf{#1}}}
\newcommand{\BuiltInTok}[1]{#1}
\newcommand{\ExtensionTok}[1]{#1}
\newcommand{\PreprocessorTok}[1]{\textcolor[rgb]{0.56,0.35,0.01}{\textit{#1}}}
\newcommand{\AttributeTok}[1]{\textcolor[rgb]{0.77,0.63,0.00}{#1}}
\newcommand{\RegionMarkerTok}[1]{#1}
\newcommand{\InformationTok}[1]{\textcolor[rgb]{0.56,0.35,0.01}{\textbf{\textit{#1}}}}
\newcommand{\WarningTok}[1]{\textcolor[rgb]{0.56,0.35,0.01}{\textbf{\textit{#1}}}}
\newcommand{\AlertTok}[1]{\textcolor[rgb]{0.94,0.16,0.16}{#1}}
\newcommand{\ErrorTok}[1]{\textcolor[rgb]{0.64,0.00,0.00}{\textbf{#1}}}
\newcommand{\NormalTok}[1]{#1}
\usepackage{graphicx,grffile}
\makeatletter
\def\maxwidth{\ifdim\Gin@nat@width>\linewidth\linewidth\else\Gin@nat@width\fi}
\def\maxheight{\ifdim\Gin@nat@height>\textheight\textheight\else\Gin@nat@height\fi}
\makeatother
% Scale images if necessary, so that they will not overflow the page
% margins by default, and it is still possible to overwrite the defaults
% using explicit options in \includegraphics[width, height, ...]{}
\setkeys{Gin}{width=\maxwidth,height=\maxheight,keepaspectratio}
\IfFileExists{parskip.sty}{%
\usepackage{parskip}
}{% else
\setlength{\parindent}{0pt}
\setlength{\parskip}{6pt plus 2pt minus 1pt}
}
\setlength{\emergencystretch}{3em}  % prevent overfull lines
\providecommand{\tightlist}{%
  \setlength{\itemsep}{0pt}\setlength{\parskip}{0pt}}
\setcounter{secnumdepth}{0}
% Redefines (sub)paragraphs to behave more like sections
\ifx\paragraph\undefined\else
\let\oldparagraph\paragraph
\renewcommand{\paragraph}[1]{\oldparagraph{#1}\mbox{}}
\fi
\ifx\subparagraph\undefined\else
\let\oldsubparagraph\subparagraph
\renewcommand{\subparagraph}[1]{\oldsubparagraph{#1}\mbox{}}
\fi

% set default figure placement to htbp
\makeatletter
\def\fps@figure{htbp}
\makeatother


\title{Morphology\_Markdown}
\author{Maggie P. MacPherson}
\date{March 19, 2021}

\begin{document}
\maketitle

\begin{Shaded}
\begin{Highlighting}[]
\KeywordTok{setwd}\NormalTok{(}\KeywordTok{dirname}\NormalTok{(rstudioapi}\OperatorTok{::}\KeywordTok{getActiveDocumentContext}\NormalTok{()}\OperatorTok{$}\NormalTok{path))}
\end{Highlighting}
\end{Shaded}

\section{Comparative Phylogenetic Comparisons of Tyrannus OTU
Morphometrics}\label{comparative-phylogenetic-comparisons-of-tyrannus-otu-morphometrics}

\subsection{Author: Maggie P.
MacPherson}\label{author-maggie-p.-macpherson}

\subsubsection{Copyright: This documentation is available under a
{[}Creative Commons{]} ()
licence}\label{copyright-this-documentation-is-available-under-a-creative-commons-licence}

\subsection{Modification History}\label{modification-history}

\subsubsection{\texorpdfstring{See {[}ReadMe Tyrannus morphology{]}
(\url{https://github.com/mmacphe/Tyrannus_morphology/commits/main/ReadMe_Tyrannus_morphology.Rmd})}{See {[}ReadMe Tyrannus morphology{]} (https://github.com/mmacphe/Tyrannus\_morphology/commits/main/ReadMe\_Tyrannus\_morphology.Rmd)}}\label{see-readme-tyrannus-morphology-httpsgithub.commmacphetyrannus_morphologycommitsmainreadme_tyrannus_morphology.rmd}

\subsection{Purpose}\label{purpose}

\subsubsection{Prepare a phylogeny composed of all currently recognized
Tyrannus subspecies with which to conduct comparative phylogenetic
principal components analyses and phylogenetic ANOVA to assess
morphological patterns across migratory, partially migratory, and
sedentary
lineages.}\label{prepare-a-phylogeny-composed-of-all-currently-recognized-tyrannus-subspecies-with-which-to-conduct-comparative-phylogenetic-principal-components-analyses-and-phylogenetic-anova-to-assess-morphological-patterns-across-migratory-partially-migratory-and-sedentary-lineages.}

\subsection{Preliminary Steps (not
shown)}\label{preliminary-steps-not-shown}

\subsubsection{\texorpdfstring{1. Download the Harvey et al. 2021
suboscine UCE phylogeny
\href{https://github.com/mgharvey/tyranni\#tyranni}{here}}{1. Download the Harvey et al. 2021 suboscine UCE phylogeny here}}\label{download-the-harvey-et-al.-2021-suboscine-uce-phylogeny-here}

\subsubsection{\texorpdfstring{2. Extract the Tyrannus genus. This was
the
\href{https://github.com/mmacphe/Tyrannus_morphology/blob/main/MacPherson_Tyrannus_base.tre}{MacPherson\_Tyrannus\_base.tre}
file that is used in Step 1,
below.}{2. Extract the Tyrannus genus. This was the MacPherson\_Tyrannus\_base.tre file that is used in Step 1, below.}}\label{extract-the-tyrannus-genus.-this-was-the-macpherson_tyrannus_base.tre-file-that-is-used-in-step-1-below.}

\subsection{Steps:}\label{steps}

\subsubsection{\texorpdfstring{1. Follow
\href{https://github.com/mmacphe/Tyrannus_morphology/blob/main/1_Tyrannus_phylogeny.R}{the
1\_Tyrannus\_phylogeny.R script} to add in additional Tyrannus
subspecies not present in the Harvey et al. 2021 UCE phylogeny. This
creates the
\href{https://github.com/mmacphe/Tyrannus_morphology/blob/main/Output\%20Files/Tyrannus_phylogeny.tre}{Tyrannus\_phylogeny.tre}
file used in subsequent comparative phylogenetic
analyses.}{1. Follow the 1\_Tyrannus\_phylogeny.R script to add in additional Tyrannus subspecies not present in the Harvey et al. 2021 UCE phylogeny. This creates the Tyrannus\_phylogeny.tre file used in subsequent comparative phylogenetic analyses.}}\label{follow-the-1_tyrannus_phylogeny.r-script-to-add-in-additional-tyrannus-subspecies-not-present-in-the-harvey-et-al.-2021-uce-phylogeny.-this-creates-the-tyrannus_phylogeny.tre-file-used-in-subsequent-comparative-phylogenetic-analyses.}

\subsubsection{\texorpdfstring{2. Follow
\href{https://github.com/mmacphe/Tyrannus_morphology/blob/main/2_morphometric_data_processing.R}{this
2\_morphometric\_data\_processing.R script} to process measurements from
voucher specimens into the formats needed for comparative phylogenetic
analyses. This code requires the following files as inputs:
\href{https://github.com/mmacphe/Tyrannus_morphology/blob/main/Tyrannus_voucher_table.csv}{Tyrannus\_voucher\_table.csv}
\href{https://github.com/mmacphe/Tyrannus_morphology/blob/main/Output\%20Files/Tyrannus_phylogeny.tre}{Tyrannus\_phylogeny.tre}
(used to match OTU names in phylogeny to their names in the voucher
table), and
\href{https://github.com/mmacphe/Tyrannus_morphology/blob/main/Tyrannus_subspecies_MigrationStrategies.csv}{Tyrannus\_subspecies\_MigrationStrategies.csv}
(a list of OTUs and their assignment as migratory, partially migratory
or sedentary). This code creates the following files as outputs:
\href{https://github.com/mmacphe/Tyrannus_morphology/blob/main/Output\%20Files/Tyrannus\%20morphology\%20data.csv}{Tyrannus
morphology data.csv} (morphometrics for all individuals along with names
that match up with the names on the phylogeny),
\href{https://github.com/mmacphe/Tyrannus_morphology/blob/main/Output\%20Files/morphology_summary_table.csv}{morphology\_summary\_table.csv}
(the mean ± standard deviation and sample size of each morphometric for
each OTU),
\href{https://github.com/mmacphe/Tyrannus_morphology/blob/main/Output\%20Files/Tyrannus_data.csv}{Tyrannus\_data.csv}
(the mean morphometric measurement for each OTU with names that match
the names in the phylogeny and movement behaviour), and
\href{https://github.com/mmacphe/Tyrannus_morphology/blob/main/Output\%20Files/cv_summary_table.csv}{cv\_summary\_table.csv}
(the coefficient of variation for each morphometric for each OTU with
names that match the names in the phylogeny and movement
behaviour).}{2. Follow this 2\_morphometric\_data\_processing.R script to process measurements from voucher specimens into the formats needed for comparative phylogenetic analyses. This code requires the following files as inputs: Tyrannus\_voucher\_table.csv Tyrannus\_phylogeny.tre (used to match OTU names in phylogeny to their names in the voucher table), and Tyrannus\_subspecies\_MigrationStrategies.csv (a list of OTUs and their assignment as migratory, partially migratory or sedentary). This code creates the following files as outputs: Tyrannus morphology data.csv (morphometrics for all individuals along with names that match up with the names on the phylogeny), morphology\_summary\_table.csv (the mean ± standard deviation and sample size of each morphometric for each OTU), Tyrannus\_data.csv (the mean morphometric measurement for each OTU with names that match the names in the phylogeny and movement behaviour), and cv\_summary\_table.csv (the coefficient of variation for each morphometric for each OTU with names that match the names in the phylogeny and movement behaviour).}}\label{follow-this-2_morphometric_data_processing.r-script-to-process-measurements-from-voucher-specimens-into-the-formats-needed-for-comparative-phylogenetic-analyses.-this-code-requires-the-following-files-as-inputs-tyrannus_voucher_table.csv-tyrannus_phylogeny.tre-used-to-match-otu-names-in-phylogeny-to-their-names-in-the-voucher-table-and-tyrannus_subspecies_migrationstrategies.csv-a-list-of-otus-and-their-assignment-as-migratory-partially-migratory-or-sedentary.-this-code-creates-the-following-files-as-outputs-tyrannus-morphology-data.csv-morphometrics-for-all-individuals-along-with-names-that-match-up-with-the-names-on-the-phylogeny-morphology_summary_table.csv-the-mean-standard-deviation-and-sample-size-of-each-morphometric-for-each-otu-tyrannus_data.csv-the-mean-morphometric-measurement-for-each-otu-with-names-that-match-the-names-in-the-phylogeny-and-movement-behaviour-and-cv_summary_table.csv-the-coefficient-of-variation-for-each-morphometric-for-each-otu-with-names-that-match-the-names-in-the-phylogeny-and-movement-behaviour.}

\subsubsection{\texorpdfstring{3. Follow
\href{https://github.com/mmacphe/Tyrannus_morphology/blob/main/3_phylogenetic_PCA.R}{this
3\_phylogenetic\_PCA.R} to conduct phylogenetic principal components
analysis of bill and feather morphometrics, and build a figure showing
the phylogeny, ancestral state reconstruction of movement ecology and
Bill PPC2, and PPCA plots for bills and 2 PPCA plots for feathers (one
with all individuals, and one zoomed in to see the trend excluding most
of the individuals from long-tailed OTUs). This code requires the
following file as inputs:
\href{https://github.com/mmacphe/Tyrannus_morphology/blob/main/Output\%20Files/Tyrannus_data.csv}{Tyrannus\_data.csv},
\href{https://github.com/mmacphe/Tyrannus_morphology/blob/main/Output\%20Files/Tyrannus_phylogeny.tre}{Tyrannus\_phylogeny.tre},
\href{https://github.com/mmacphe/Tyrannus_morphology/blob/main/Output\%20Files/Tyrannus\%20morphology\%20data.csv}{Tyrannus
morphology data.csv},
\href{https://github.com/mmacphe/Tyrannus_morphology/blob/main/Output\%20Files/cv_summary_table.csv}{cv\_summary\_table.csv},
\href{https://github.com/mmacphe/Tyrannus_morphology/blob/main/Output\%20Files/Tyrannus\%20morphology\%20\%2B\%20PCA\%20avg.csv}{Tyrannus
morphology + PCA avg.csv} (made during this part of the pipeline and
used for th ancestral state reconstruction for the figure), and
\href{https://github.com/mmacphe/Tyrannus_morphology/blob/main/Tyrannus_subspecies_MigrationStrategies.csv}{Tyrannus\_subspecies\_MigrationStrategies.csv}.
This code creates the following files as outputs:
\href{https://github.com/mmacphe/Tyrannus_morphology/blob/main/Output\%20Files/Tyrannus\%20morphology\%20\%2B\%20PCA\%20avg.csv}{Tyrannus
morphology + PCA avg.csv},
\href{https://github.com/mmacphe/Tyrannus_morphology/blob/main/Output\%20Files/cv_summary.csv}{cv\_summary.csv},
\href{https://github.com/mmacphe/Tyrannus_morphology/blob/main/Output\%20Files/Tyrannus_phylogenetic_PCA.png}{Tyrannus\_phylogenetic\_PCA.png}.}{3. Follow this 3\_phylogenetic\_PCA.R to conduct phylogenetic principal components analysis of bill and feather morphometrics, and build a figure showing the phylogeny, ancestral state reconstruction of movement ecology and Bill PPC2, and PPCA plots for bills and 2 PPCA plots for feathers (one with all individuals, and one zoomed in to see the trend excluding most of the individuals from long-tailed OTUs). This code requires the following file as inputs: Tyrannus\_data.csv, Tyrannus\_phylogeny.tre, Tyrannus morphology data.csv, cv\_summary\_table.csv, Tyrannus morphology + PCA avg.csv (made during this part of the pipeline and used for th ancestral state reconstruction for the figure), and Tyrannus\_subspecies\_MigrationStrategies.csv. This code creates the following files as outputs: Tyrannus morphology + PCA avg.csv, cv\_summary.csv, Tyrannus\_phylogenetic\_PCA.png.}}\label{follow-this-3_phylogenetic_pca.r-to-conduct-phylogenetic-principal-components-analysis-of-bill-and-feather-morphometrics-and-build-a-figure-showing-the-phylogeny-ancestral-state-reconstruction-of-movement-ecology-and-bill-ppc2-and-ppca-plots-for-bills-and-2-ppca-plots-for-feathers-one-with-all-individuals-and-one-zoomed-in-to-see-the-trend-excluding-most-of-the-individuals-from-long-tailed-otus.-this-code-requires-the-following-file-as-inputs-tyrannus_data.csv-tyrannus_phylogeny.tre-tyrannus-morphology-data.csv-cv_summary_table.csv-tyrannus-morphology-pca-avg.csv-made-during-this-part-of-the-pipeline-and-used-for-th-ancestral-state-reconstruction-for-the-figure-and-tyrannus_subspecies_migrationstrategies.csv.-this-code-creates-the-following-files-as-outputs-tyrannus-morphology-pca-avg.csv-cv_summary.csv-tyrannus_phylogenetic_pca.png.}

\subsubsection{\texorpdfstring{4. Follow
\href{https://github.com/mmacphe/Tyrannus_morphology/blob/main/4_phyloANOVA.R}{this
4\_phyloANOVA.R} to conduct the appropriate phylogenetic ANOVAs for each
part of the dataset: morphometrics while accounting for body size (using
tarsus length), PPCA scores, and coefficients of variation. This code
requires the following files as inputs:
\href{https://github.com/mmacphe/Tyrannus_morphology/blob/main/Output\%20Files/Tyrannus_phylogeny.tre}{Tyrannus\_phylogeny.tre},
\href{https://github.com/mmacphe/Tyrannus_morphology/blob/main/Output\%20Files/Tyrannus\%20morphology\%20\%2B\%20PCA\%20avg.csv}{Tyrannus
morphology + PCA avg.csv}, and
\href{https://github.com/mmacphe/Tyrannus_morphology/blob/main/Output\%20Files/cv_summary.csv}{cv\_summary.csv}.
This code creates the following code as outputs:
\href{https://github.com/mmacphe/Tyrannus_morphology/blob/main/Output\%20Files/phylANOVA_output.txt}{phylANOVA\_output.txt},
\href{https://github.com/mmacphe/Tyrannus_morphology/blob/main/Output\%20Files/phylANOVA_output_PCscores.txt}{phylANOVA\_output\_PCscores.txt},
\href{https://github.com/mmacphe/Tyrannus_morphology/blob/main/Output\%20Files/phylANOVA_cv_output.txt}{phylANOVA\_cv\_output.txt},
and
\href{https://github.com/mmacphe/Tyrannus_morphology/blob/main/Output\%20Files/phylANOVA_tarsus-corrected_residuals.csv}{phylANOVA\_tarsus-corrected\_residuals.csv}
(used to build boxplots in
\href{https://github.com/mmacphe/Tyrannus_morphology/blob/main/Boxplots_Figures.R}{Boxplots\_Figures.R}).}{4. Follow this 4\_phyloANOVA.R to conduct the appropriate phylogenetic ANOVAs for each part of the dataset: morphometrics while accounting for body size (using tarsus length), PPCA scores, and coefficients of variation. This code requires the following files as inputs: Tyrannus\_phylogeny.tre, Tyrannus morphology + PCA avg.csv, and cv\_summary.csv. This code creates the following code as outputs: phylANOVA\_output.txt, phylANOVA\_output\_PCscores.txt, phylANOVA\_cv\_output.txt, and phylANOVA\_tarsus-corrected\_residuals.csv (used to build boxplots in Boxplots\_Figures.R).}}\label{follow-this-4_phyloanova.r-to-conduct-the-appropriate-phylogenetic-anovas-for-each-part-of-the-dataset-morphometrics-while-accounting-for-body-size-using-tarsus-length-ppca-scores-and-coefficients-of-variation.-this-code-requires-the-following-files-as-inputs-tyrannus_phylogeny.tre-tyrannus-morphology-pca-avg.csv-and-cv_summary.csv.-this-code-creates-the-following-code-as-outputs-phylanova_output.txt-phylanova_output_pcscores.txt-phylanova_cv_output.txt-and-phylanova_tarsus-corrected_residuals.csv-used-to-build-boxplots-in-boxplots_figures.r.}

\subsection{Also included:}\label{also-included}

\subsubsection{\texorpdfstring{1. .R code to produce our boxplot figures
\href{https://github.com/mmacphe/Tyrannus_morphology/blob/main/Boxplots_Figures.R}{here}.}{1. .R code to produce our boxplot figures here.}}\label{r-code-to-produce-our-boxplot-figures-here.}

\subsubsection{\texorpdfstring{2. .R code to assess age and sex class
influences on mean morphometric values
\href{https://github.com/mmacphe/Tyrannus_morphology/blob/main/LinearModels_Demographic_Influence.R}{here}.}{2. .R code to assess age and sex class influences on mean morphometric values here.}}\label{r-code-to-assess-age-and-sex-class-influences-on-mean-morphometric-values-here.}

\subsection{Other things to note before diving right in to using this
code:}\label{other-things-to-note-before-diving-right-in-to-using-this-code}

\subsubsection{\texorpdfstring{1. The .R files are all set up so that
new runs save outputs to the source folder. All files created by the
author are found in the
\href{https://github.com/mmacphe/Tyrannus_morphology/tree/main/Output\%20Files}{Output
Files
folder}.}{1. The .R files are all set up so that new runs save outputs to the source folder. All files created by the author are found in the Output Files folder.}}\label{the-.r-files-are-all-set-up-so-that-new-runs-save-outputs-to-the-source-folder.-all-files-created-by-the-author-are-found-in-the-output-files-folder.}

\subsubsection{2. Code to make separate output files, conduct separate
analyses and create separate figures for females and males are commented
out in the .R code files. Follow the instructions in each .R file to
exchange lines of code with the sex-specific lines of code for analyses
looking at sexes
differently.}\label{code-to-make-separate-output-files-conduct-separate-analyses-and-create-separate-figures-for-females-and-males-are-commented-out-in-the-.r-code-files.-follow-the-instructions-in-each-.r-file-to-exchange-lines-of-code-with-the-sex-specific-lines-of-code-for-analyses-looking-at-sexes-differently.}

\section{Enjoy!}\label{enjoy}

\end{document}
